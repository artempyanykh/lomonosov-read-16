\documentclass{beamer}

\usepackage{pscyr}

\usepackage[T2A]{fontenc}
\usepackage[utf8]{inputenc}
\usepackage[english, russian]{babel}

\usepackage{amsmath}
\usepackage{amssymb}
\usepackage{amsthm}

\title[Биржевые торги с непрерывными ставками]{%
  О модификации многошаговой модели биржевых торгов с непрерывными ставками%
}

\author{%
Пьяных А.И.\\
\texttt{artem.pyanykh@gmail.com}%
}

\institute{%
  Московский Государственный Университет \\
  Факультет вычислительной математики и кибернетики%
}

\date{Ломоносовские чтения. 2016}

\subject{Game theory}

\usetheme{Warsaw}

\begin{document}

\begin{frame}
  \titlepage
\end{frame}

\begin{frame}
  \frametitle{Содержание}
  \tableofcontents
\end{frame}

\section[]{Краткое описание модели}

\begin{frame}
  Модель
\end{frame}

\section[]{Обзор существующих результатов}

\begin{frame}
  Существующие результаты
\end{frame}

\section[]{Постановка задачи}

\begin{frame}
  Постановка задачи
\end{frame}

\section[]{Анализ прямой и двойственной игр}

\begin{frame}
  Анализ прямой и двойственной игр
\end{frame}

\section[]{Сравнение результатов}

\begin{frame}
  Сравнение результатов
\end{frame}

\end{document}
