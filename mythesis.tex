\documentclass{lra}

\begin{abstract}
  \Title{О модификации многошаговой модели биржевых торгов с непрерывными
    ставками}%
  {Пьяных А.И.}

  \OneAuthor{Пьяных А.И.}%
  {Кафедра исследования операций}{artem.pyanykh@gmail.com}

  % Исследуется модификация многошаговой модели биржевых торгов, рассмотренной в
  % [\ref{pyanykh_demeyer02}]. Два игрока ведут торги за однотипные акции,
  % ликвидная цена которых определяется перед началом торгов и может принимать два
  % значения: $1$ и $0$ с вероятностями $p$ и $1-p$ соответственно. Игрок 1
  % (инсайдер) осведомлен о реальном значении цены, в то время как игроку 2
  % известно только вероятностное распределение. На каждом шаге торгов игроки
  % делают вещественные ставки. Игрок, предложивший большую ставку, покупает у
  % другого акцию.
  
  % В работе [\ref{pyanykh_demeyer02}] цена сделки равна наибольшей предложенной
  % ставке

  Рассматривается модель финансового рынка, в которой два игрока ведут торги за
  однотипные акции в течение $n$ шагов. Цена акции может принимать значения $0$
  и $1$ с вероятностями $1-P$ и $P$. Игрок 1 информирован о настоящей цене,
  игрок 2 знает только вероятностное распределение. На каждом шаге торгов игроки
  делают ставки, причем игрок, предложивший б\'{о}льшую ставку, покупает у другого
  акцию. Обозначим $p_{max}, \, p_{min}$ большую и меньшую ставки.

  Модели с дискретными и непрерывными ставками рассматривались в
  [\ref{pyanykh_domansky07}] и [\ref{pyanykh_demeyer02}] соответственно. В обеих
  работах сделка осуществляется по цене $p_{max}$. Обобщение дискретной модели
  на случай продажи акции по цене $\beta p_{max} + (1-\beta) p_{min}, \, \beta
  \in [0,1]$ рассмотрено в [\ref{pyanykh_pyanykh16}].
  
  В данной работе получено обобщение для модели с непрерывными ставками: найдены
  оптимальные стратегии игроков и значение $V_n(P)$ соответствующей $n$-шаговой
  игры.
  %
  Пусть $\lambda = V'_n(P)$ и $W_n(x) = \mathbb{E} \left[ \min(x - \sum_{i=1}^n
    U_i, 0)\right]$, где $U_i$ равномерно распределены на $[-1,1]$. Тогда
  оптимальная стратегия игрока 1 задается функциями
  \begin{equation*}
    Q(u) = W'_{n-1}(\lambda+1-2u), \;
    f(u) = \frac{\int_{1-\beta}^u 2 (x-1+\beta) Q(x) dx}{(u-1+\beta)^2}.
  \end{equation*}

  Отметим, что хотя оптимальные стратегии игроков зависят от $\beta$, значение
  игры неизменно и совпадает с таковым в [\ref{pyanykh_demeyer02}].

%   Это --- пример оформления файла тезисов. Свой тезис можно делать, редактируя
%   исходный \texttt{tex}-файл этого примера в соответствующей кодировке. Для
%   самих правил оформления тезисов обратитесь к {\ttfamily instructions.pdf}.

%   Формулы можно объявлять привычным образом:
%   \begin{equation}\label{obrazcov_main_eq}
%     \int\limits_a^b f'(x)dx = f(b) - f(a),
%   \end{equation}
%   а делать ссылки на них --- с момощью команды \texttt{eqref}:
%   \eqref{obrazcov_main_eq}.

%   Для оформления теорем, лемм и определений рекомендуется использовать
%   предоставляемые стилевым файлом окружения:
%   \begin{theorem}\label{obrazcov_theorem}
%     Соотношение \eqref{obrazcov_main_eq} является частчным случаем формулы
%     Стокса.
%   \end{theorem}

% \begin{lemma}
%   Текст леммы ссылается на Теорему~\ref{obrazcov_theorem}.
% \end{lemma}

% \begin{definition}
%   Будем называть квадратную матрицу $A$ положительно определённой, если $x'Ax >
%   0$ для любого $x\neq 0$.
% \end{definition}

% При подготовке иллюстраций стоит помнить, что сборник будет напечатан
% черно-белым в формате А5.

% Убедительная просьба следить за тем, чтобы объем тезиса не превышал одной
% страницы. Настоящий пример нарушает это требование для того, чтобы вместить в
% себя больше примеров оформления элементов тезиса.

%
%	Если благодарность выражать не надо, то следущий код нужно убрать.
%
% \Gratitude Работы поддержана грантом РФФИ \No 3141--5926--53--58.

%
%	Библиография создается вручную, заголовок дается командой \References
%
\References
\begin{references}
\item\label{pyanykh_domansky07}%
  \ENGLISH{Domansky V. Repeated games with asymmetric information and random
    price fluctuations at finance markets // Int J Game Theory. 2007. V. 36(2).
    P. 241--257.}
\item\label{pyanykh_demeyer02}%
  \ENGLISH{De Meyer B., Saley H. On the strategic origin of Brownian motion in
    finance // Int J Game Theory. 2002. V. 31. P. 285--319.}
\item\label{pyanykh_pyanykh16}%
  Пьяных А.И. Многошаговая модель биржевых торгов с асимметричной информацией и
  элементами переговоров // Вестн. Моск. ун-та. Сер.15. Вычисл. матем. и киберн.
  2016. \No 1. С. 34--40.
\end{references}

\end{abstract}
