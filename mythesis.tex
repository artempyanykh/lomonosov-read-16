\documentclass{lra}

\begin{abstract}
\Title{Об одном свойстве одной задачи}{Образцов И.И., Примеров П.П.}
% Примеры:
%\OneAuthor{Образцов Иван Иванович}{Кафедра примеров и образцов}{i.i.example@gmail.com}
%

\TwoAuthor{Образцов Иван Иванович}{Кафедра примеров и образцов}{i.i.example@gmail.com}{Примеров Петр Петрович}{Кафедра шаблонов и трафаретов}{p.p.example@mail.ru}

%\ThreeAuthor{Образцов Иван Иванович}{Кафедра примеров и образцов}{i.i.example@gmail.com}{Примеров Петр Петрович}{Кафедра шаблонов и трафаретов}{p.p.example@mail.ru}{Трафаретов Дмитрий Дмитриевич}{Кафедра паттернов и парадигм}{d.d.example@gmail.com}
%


Это --- пример оформления файла тезисов. Свой тезис можно делать, редактируя исходный \texttt{tex}-файл этого примера в соответствующей кодировке.
Для самих правил оформления тезисов обратитесь к {\ttfamily instructions.pdf}.

Формулы можно объявлять привычным образом:
\begin{equation}\label{obrazcov_main_eq}
 \int\limits_a^b f'(x)dx = f(b) - f(a),
\end{equation}
а делать ссылки на них --- с момощью команды \texttt{eqref}: \eqref{obrazcov_main_eq}.

Для оформления теорем, лемм и определений рекомендуется использовать предоставляемые стилевым файлом окружения:
\begin{theorem}\label{obrazcov_theorem}
 Соотношение \eqref{obrazcov_main_eq} является частчным случаем формулы Стокса.
\end{theorem}

\begin{lemma}
 Текст леммы ссылается на Теорему \ref{obrazcov_theorem}.
\end{lemma}

\begin{definition}
Будем называть квадратную матрицу $A$  положительно определённой, если $x'Ax > 0$ для любого $x\neq 0$. 
\end{definition}

При подготовке иллюстраций стоит помнить, что сборник будет напечатан черно-белым в формате А5.

Убедительная просьба следить за тем, чтобы объем тезиса не превышал одной страницы. Настоящий пример нарушает это требование для того, чтобы вместить в себя больше примеров оформления элементов тезиса.

%
%	Если благодарность выражать не надо, то следущий код нужно убрать.
%
\Gratitude
Работы поддержана грантом РФФИ \No 3141--5926--53--58.

%
%	Библиография создается вручную, заголовок дается командой \References
%
\References
\begin{references}
 \item Тихонов А.Н. Об устойчивости обратных задач // Доклады АН СССР, Т. 39, \No 5, С. 195--198.
 \item Львовский С.М. Набор и верстка в \LaTeX. М.: МЦНМО, 2003.
 \item \ENGLISH{Feller W. An Introduction to Probability Theory and its Applications, Volume I. 3rd edition. New York: Wiley, 1968.}
 \item Иванов И.И., Петров П.П. О некоторых вопросах оформления списков литературы // Труды конференции <<Стандартизация>>, Москва, 2016, С. 23--25
 \item Официальный сайт факультета ВМК: \url{cs.msu.ru}
\end{references}

%
%	Если иллюстраций нет, то последующий код нужно убрать.
% 

\Pictures
\TwoPic{5cm}{obrazcov_pic_1.eps}{obrazcov_pic_2.eps}{Рисунок 1: Подпись к картинке.}

\end{abstract}